%-------------------------
% Resume in Latex
% Original Author : Sourabh Bajaj
% Adaptation : Hyunggi Chang
% License : MIT
%------------------------

\documentclass[letterpaper,11pt]{article}

\usepackage{latexsym}
\usepackage[empty]{fullpage}
\usepackage{titlesec}
\usepackage{marvosym}
\usepackage[usenames,dvipsnames]{color}
\usepackage{verbatim}
\usepackage{enumitem}
\usepackage[hidelinks]{hyperref}
\usepackage{fancyhdr}
\usepackage[english]{babel}
\usepackage{tabularx}
\usepackage{amsmath}
\usepackage{kotex} % Enable Korean!

\pagestyle{fancy}
\fancyhf{} % clear all header and footer fields
\fancyfoot{}
\renewcommand{\headrulewidth}{0pt}
\renewcommand{\footrulewidth}{0pt}

% Adjust margins
\addtolength{\oddsidemargin}{-0.5in}
\addtolength{\evensidemargin}{-0.5in}
\addtolength{\textwidth}{1in}
\addtolength{\topmargin}{-0.5in}
\addtolength{\textheight}{1.0in}

\urlstyle{same}

\raggedbottom
\raggedright
\setlength{\tabcolsep}{0in}

% Sections formatting
\titleformat{\section}{
  \vspace{-4pt}\scshape\raggedright\large
}{}{0em}{}[\color{black}\titlerule \vspace{-2pt}]

%-------------------------
% Custom commands
\newcommand{\resumeItem}[1]{
  \item\small{
    {#1 \vspace{-2pt}}
  }
}

\newcommand{\resumeSummary}[1]{
  \item
    \begin{tabular*}{0.97\textwidth}[t]{l@{\extracolsep{\fill}}r}
      #1
    \end{tabular*}
}

\newcommand{\resumeSubheading}[4]{
  \vspace{-1pt}\item
    \begin{tabular*}{0.97\textwidth}[t]{l@{\extracolsep{\fill}}r}
      \textbf{#1} & #2 \\
      \textit{\small#3} & \textit{\small #4} \\
    \end{tabular*}\vspace{-5pt}
}

\newcommand{\resumeEmployment}[4]{
  \vspace{-1pt}\item
    \begin{tabular*}{0.97\textwidth}[t]{l@{\extracolsep{\fill}}r}
      \textbf{#1} & #2 \\
      \textit{\small#3} & \textit{\small #4} \\
    \end{tabular*}\vspace{-5pt}
}

\newcommand{\resumeProject}[2]{
  \vspace{-1pt}\item
    \begin{tabular*}{0.97\textwidth}[t]{l@{\extracolsep{\fill}}r}
      \textbf{#1} \\
      \small{#2} \\
    \end{tabular*}\vspace{-5pt}
}

\newcommand{\resumeResearch}[5]{
  \vspace{-1pt}\item
    \begin{tabular*}{0.97\textwidth}[t]{l@{\extracolsep{\fill}}r}
      \textbf{#1} & #2 \\
      \textit{\small#3} {\small #4 \vspace{-2pt}} & \textit{\small #5} \\
    \end{tabular*}\vspace{-5pt}
}

\newcommand{\resumeTalk}[2]{
  \vspace{-1pt}\item
    \begin{tabular*}{0.97\textwidth}[t]{l@{\extracolsep{\fill}}r}
      \textbf{#1} & #2 \\
    \end{tabular*}\vspace{-5pt}
}

\newcommand{\resumeSkills}[1]{
  \item
    \begin{tabular*}{0.97\textwidth}[t]{l@{\extracolsep{\fill}}r}
      #1
    \end{tabular*}
}

\newcommand{\resumeCommunity}[3]{
  \vspace{-1pt}\item
    \begin{tabular*}{0.97\textwidth}[t]{l@{\extracolsep{\fill}}r}
      \textbf{#1} & #2 \\
      \textit{\small#3} \\
    \end{tabular*}\vspace{-5pt}
}

\newcommand{\resumeSubItem}[2]{\resumeItem{#1}{#2}\vspace{-4pt}}

\renewcommand{\labelitemii}{$\circ$}

\newcommand{\resumeSubHeadingListStart}{\begin{itemize}[leftmargin=*]}
\newcommand{\resumeSubHeadingListEnd}{\end{itemize}}

\newcommand{\resumeEmploymentListStart}{\begin{itemize}[leftmargin=*]}
\newcommand{\resumeEmploymentListEnd}{\end{itemize}}

\newcommand{\resumeItemListStart}{\begin{itemize}}
\newcommand{\resumeItemListEnd}{\end{itemize}\vspace{-5pt}}

%-------------------------------------------
%%%%%%  CV STARTS HERE  %%%%%%%%%%%%%%%%%%%%%%%%%%%%


\begin{document}

%----------HEADING-----------------
\begin{tabular*}{\textwidth}{l@{\extracolsep{\fill}}r}
  \textbf{\Large 김솔지}} &  \\
  \href{https://github.com/so1gging}{Github} $|$ \href{https://www.linkedin.com/in/solzykim/}{LinkedIn} & Email : \href{mailto:YourEmail@gmail.com}{yzlosmik@gmail.com} \\
  {} & Mobile : (+82) 010-4854-7484 \\
\end{tabular*}

%-----------Summary-----------------
    \resumeSummary{
    안녕하세요, 6년 차 프론트엔드 개발자입니다. \\ 
소통으로 더 나은 사용자 경험과 제품 완성도를 만드는 데 가치를 둡니다.\\ 팀원들과 함께 고민하고 성장하는 문화를 중요하게 생각합니다.}

%-----------EXPERIENCE-----------------
\section{경력}
\resumeEmploymentListStart

\resumeEmployment
  {알스퀘어}{2021.09 -- 재직중}
  {분석 솔루션, 지도 기반 시각화, 백오피스 등 다양한 도메인의 프로젝트를 수행했습니다.}{}
  \vspace{0.05em}\\
  \textit {\small 테크리드로서 기술적 의사결정과 품질 관리를 주도한 경험이 있습니다.}
  \vspace{0.05em}\\
  \textit {\small PoC → 출시 → 운영까지 서비스 전 생명주기 경험했습니다.}

% --- Project 1 ---
\vspace{0.6em}
{\normalsize{실거래 데이터 기반 분석 솔루션 - }\href{https://www.rsquare.co.kr/rsquare-analytics}{\color{blue} Rsquare Analytics}.}
\hfill {\small 2025.08 -- 현재}
\resumeItemListStart
  \resumeItem{\textbf{인터랙션 개선}을 통해 서버 트래픽과 응답 지연 문제 완화}
  \resumeItem{서버 상태 라이브러리를 활용해 \textbf{UX 응답성 개선}}
  \resumeItem{\textbf{E2E 테스트 도입}으로 QA 자동화 및 배포 안정성 향상}
  \resumeItem{\textbf{CI/CD 파이프라인 개선}으로 평균 배포 시간 50\% 이상 단축}
\resumeItemListEnd

% --- Project 2 ---
\vspace{0.6em}
{\normalsize{사내 백오피스 서비스}}
\hfill {\small 2024.07 -- 2025.07}
\resumeItemListStart
    \resumeItem{\textbf{프론트엔드 테크리드}로서 팀원 기술적 방향 제시 및 프로젝트 운영}
    \resumeItem{React Query의 캐싱 전략을 활용해 불필요한 API 호출을 줄이고 응답 성능 개선}
    \resumeItem{누적된 운영 업무의 70\% 이상을 해결하여 서비스 안정성과 사용자 만족도 향상}
    \resumeItem{기존 E2E 테스트 환경 Playwright로 마이그레이션 및 속도 개선}
\resumeItemListEnd

% --- Project 3 ---
\vspace{0.6em}
{\normalsize{지도 기반 데이터 시각화 서비스}}
\hfill {\small 2021.12 -- 2024.07, 재참여}
\resumeItemListStart
    \resumeItem{신규 프로젝트 PoC부터 출시 전 과정 프론트엔드 개발 참여}
    \resumeItem{JavaScript 기반 Naver Map API를 선언적으로 구현하여 \textbf{사내 라이브러리화}}
    \resumeItem{canvas 기반 맵 레이어를 구현해 수만 건의 DOM 엘리먼트를 효율적으로 시각화하여 성능 개선}
    \resumeItem{정적 분석 도구 SonarQube를 활용하여 버그를 사전식별하고 코드 품질을 향상 기여}
\resumeItemListEnd

% --- Project 4 ---
\vspace{0.6em}
{\normalsize{사내 프로젝트 개발 및 유지보수 담당}}
\hfill {\small 2020.09 -- 2021.12}
\resumeItemListStart
    \resumeItem{웹/앱 화면 개발 및 api 개발}
    \resumeItem{데스크탑과 모바일에 최적화된 반응형 개발}
    \resumeItem{Android 기반 APP 유지보수}
    \resumeItem{AWS SNS 서비스를 연동하여 글로벌 문자인증 서비스 제공}
    \resumeItem{firebase notification service를 이용해 app 메세징 서비스 개발}
\resumeItemListEnd

\resumeEmploymentListEnd

%-----------EDUCATION-----------------
\section{교육}
  \resumeSubHeadingListStart
    \resumeSubheading
      {강남대학교}{Young-in, South Korea}
      {컴퓨터공학과 학사}{2016.03 -- 2020.02}
  \resumeSubHeadingListEnd
    
%-----------Papers-----------------
\section{자격}
    \resumeSubHeadingListStart
        \resumeItem{\textbf{정보처리기사} 2019.11}
        \resumeItem{\textbf{컴퓨터활용능력 1급} 2019.09}
    \resumeSubHeadingListEnd

\newpage

%-----------자기소개서-----------------
\section{자기소개서}

안녕하세요. 6년 차 프론트엔드 개발자 김솔지입니다.

\vspace{0.6em}

2020년부터 프롭테크 기업에서 근무하며 상업용 부동산 데이터를 기반으로 한 다양한 프로젝트를 담당해 왔습니다.
주로 지도 기반 데이터 시각화 서비스를 중심으로, 요구사항에 맞는 시스템을 설계하고 구현해 왔으며
시각화 중심 프로젝트의 특성상 성능 최적화에도 지속적으로 관심을 가지고 문제를 주도적으로 해결해 왔습니다.
또한 기본적인 CRUD API 및 쿼리 개발까지 경험하며, 프론트엔드와 백엔드 전반에 걸친 협업을 자연스럽게 익혔습니다.

\vspace{0.6em}

저는 프론트엔드 개발을 단순히 화면을 구성하는 작업이라고 생각하지 않습니다.
프론트엔드는 사용자의 요구사항을 가장 빠르게 반영하고,
서비스의 사용자 경험을 실질적으로 개선해 나가는 역할이라고 생각합니다.
항상 사용자의 입장에서 고민하며 요구사항을 적극적으로 듣고 이해하고,
더 나은 방향이 있다면 제안하는 개발자가 되기 위해 노력해 왔습니다.

\vspace{0.6em}

개발 과정에서 제가 가장 중요하게 생각하는 가치는 \textbf{협업}입니다.
개인의 완성도보다는 팀원들과 함께 문제를 정의하고 해결해 나가는 과정을 중요하게 여기며,
자유롭게 의견을 나눌 수 있는 소통 중심의 개발 문화를 선호합니다.
이러한 가치관을 바탕으로 약 1년간 Tech Leader로서 여러 소규모 프로젝트를 리딩하며
팀과 함께 성장하는 경험을 쌓아왔습니다.

\vspace{0.6em}

지금까지의 경험을 바탕으로,
앞으로 함께할 동료들과 제품에 지속적으로 긍정적인 가치를 더할 수 있는 개발자가 되고자 합니다.

%-------------------------------------------
\end{document}
